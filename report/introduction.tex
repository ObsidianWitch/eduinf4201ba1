\section{Introduction}
L'objectif de ce TP est de se familiariser avec les sockets et comment faire communiquer plusieurs applications entre-elles sur un réseau. Deux types de socket seront utilisés, les sockets non connectées (\emph{SOCK\_DGRAM}) et les sockets connectées (\emph{SOCK\_STREAM}). De plus, le protocole HTTP sera abordé à travers la réalisation d'un client émettant une requête et recevant la réponse correspondante, et la création d'un serveur web répondant à des requêtes.

\subsection{Arborescence}
\noindent Ci-dessous l'arborescence de fichiers du TP avec pour certains d'entre-eux une brève description.

\dirtree{%
.1 src.
.2 exo1.
.3 client.c.
.3 server.c.
.3 msg\_tools.c/.h - Création et extraction d'informations de messages..
.2 exo2.
.3 client.c.
.3 server.c.
.2 exo2\_alt.
.3 client.c.
.2 exo3.
.3 client.c.
.2 exo4.
.3 server.c.
.2 exo5.
.3 client.c.
.2 file\_tools.c/.h - Manipulation de fichiers (transfert de données entre\\ descripteurs de fichiers)..
.2 http\_tools.c/.h - Manipulation de requêtes HTTP (création, réception, parsing,\\ affichage de requêtes)..
.2 socket\_tools.c/.h - Manipulation de sockets (initialisation de sockets, envoi\\ complet d'un message, réception d'un message)..
}

\subsection{Instructions de compilation}
Ci-dessous les instructions pour compiler l'ensemble des exercices. L'utilitaire cmake \cite{cite:cmake} est nécessaire afin de créer le Makefile.

\begin{mdframed}[backgroundcolor=lightblue, linecolor=darkblue]
	mkdir build\\
	cd build\\
	cmake ..\\
	make
\end{mdframed}
